% !TEX root = ../Thesis.tex


\chapter{Introduzione}

Le vespe appartenenti alla famiglia Vespidae rappresentano un gruppo di insetti sociali di notevole interesse sia ecologico che chimico-biologico, grazie alla complessità dei materiali che impiegano nella costruzione dei propri nidi \cite{Cole2001,Jeanne}.
Questi materiali, derivati principalmente da fonti vegetali, come le fibre di legno, e da secrezioni di origine animale, come la saliva (secrezioni orali proteiche), costituiscono biocompositi naturali nei quali componenti cellulosiche e proteiche interagiscono conferendo al nido specifiche proprietà meccaniche, strutturali e di resistenza ambientale, inclusa l'impermeabilizzazione \cite{Curtis,Singer1992}.
Lo studio della composizione chimica e strutturale dei nidi consente di comprendere non solo le strategie costruttive e adattative delle diverse specie di vespe, ma anche di individuare analogie con materiali polimerici naturali di potenziale interesse per la scienza dei materiali, fungendo da ispirazione per nuove soluzioni tecnologiche.

In questa tesi si analizzano nidi di diverse specie appartenenti alla famiglia Vespidae, con particolare attenzione alla componente cellulosica e proteica, mediante tecniche di spettroscopia NMR allo stato solido e altre analisi complementari. L’obiettivo è quello di correlare la composizione chimica dei nidi con le differenti modalità costruttive e ambientali delle specie considerate, contribuendo alla comprensione dei processi biochimici e strutturali alla base della formazione di questi peculiari biocompositi naturali.

\section{La famiglia Vespidae}


La famiglia Vespidae rappresenta un gruppo estremamente diversificato di Imenotteri, con una vasta gamma di abitudini sociali e morfologiche, ed è distribuita in diverse sottofamiglie, tra cui le eusociali Polistinae e Vespinae. \cite{Jeanne}
Le specie eusociali si caratterizzano per una distinta divisione in caste (regina, operai e maschi), tuttavia si osservano comunque profonde differenze riguardo le strategie adottate per la nidificazione e nella diversa adattabilità ambientale. \cite{Jeanne}

L'architettura del nido è un tratto evolutivo cruciale, plasmato da fattori ambientali e dalla forte pressione predatoria, in particolare da parte delle formiche, nelle regioni tropicali. \cite{Jeanne}
Si possono distinguere due principali strategie architettoniche, specie del genere Polistes per esempio realizzano nidi con favo scoperto sospesi tramite un sottile peduncolo difeso chimicamente con una secrezione ghiandolare repellente, questa tipologie di nidi sono detti stelocittari gimnodomi. \cite{Jeanne}
L'altra tipologia è rappresentata dai nidi caliptomi tipico dei generi Vespula e Vespa, sono nidi caratterizzati da un favo racchiuso in un involucro dove l'accesso è rappresentato da un'unica fessura o foro, nidi di questo tipo facilitando la difesa e forniscono anche isolamento termico. \cite{Jeanne,Cole2001,Schmolz}
Il materiale primario di costruzione è il cartone, che si origina dalla miscelazione di fibre vegetali (cellulose), talvolta legno alterato o triconi specifici, che ne influenzano la resistenza e la densità, con una secrezione orale proteica ricca di aminoacidi come prolina, glicina e serina, che funge da matrice adesiva, conferendo robustezza e idrorepellenza. \cite{Jeanne}
Infine, l'organizzazione comportamentale all'interno della colonia è complessa e dinamica, implicando una ripartizione dei compiti (foraggiatori d'acqua e polpa, costruttori), con la regolazione di tale ripartizione che emerge dalla disponibilità di risorse idriche temporaneamente immagazzinate nell'ingluvie delle operaie, un meccanismo noto come "stomaco comune" che funge da centro di informazione omeostatico per bilanciare le esigenze di costruzione e metaboliche della colonia. \cite{Jeanne}


\section{Genere Polistes}


Andando ad osservare nel dettaglio le abitudini di vespe del genere Polistes (in particolare Polistes Dominula e Polistes Gallicus) vedremo che questi vespidi realizzano i propri nidi con un materiale di tipo cartaceo, che viene prodotto tramite l'utilizzo di fibre vegetali, per esempio frammenti di corteccia, che vengono masticati e quindi mescolati con secrezioni salivari di natura proteica; queste hanno lo scopo li legare le fibre vegetali e formare un nido compatto, am anche quello di impermeabilizzare il nido. A livello morfologico queste specie danno origine a nidi zzz  Le Polistes (es. P. dominula, P. gallicus) costruiscono nidi aerei aperti, costituiti da un solo favo di celle esagonali sospeso tramite un peduncolo (petiolo) [8]. Questi nidi, privi di involucro esterno, sono realizzati con un materiale di tipo “cartaceo” ottenuto dalla masticazione di fibre vegetali (legno, corteccia o steli secchi) mescolate a secrezioni salivari proteiche che fungono da agente legante e impermeabilizzante [9,10].
La secrezione orale è vitale per la protezione del nido dagli agenti atmosferici. La quantità di saliva è specie-specifica, come dimostrato dalla notevole variazione tra P. nimpha (58) e le altre due specie (22-23) \cite{Bağriaçik}


\section{Genere Vespa e Vespula}


Le specie appartenenti al genere Vespa e Vespula sono vespidi sociali che si sono adattati prevalentemente alle regioni temperate. \cite{Jeanne}
Si hanno caratteristiche in comune riguardo la struttura dei nidi realizzati dalle differenti specie di questi generi, tra cui la differenziazione della struttura delle celle interne e dell'involucro esterno. \cite{Cole2001}
Questi nidi sono realizzati con un caratteristico sistema di favi multipli impilati verticalmente uno sotto l'altro e poi avvolti dall'involucro esterno. \cite{Jeanne}
L'involucro è realizzato in modo da fornire protezione alle celle e al nido nel complesso, garantendo isolamento termico e protezione dagli agenti atmosferici \cite{Cole2001},
Il nido è costruito a partire dal cartone, materiale leggero ottenuto dalla masticazione di fibre vegetali mescolate con una secrezione orale. \cite{Jeanne}
I nidi della specie Vespa crabro presentano queste esatte cartteristiche e hanno generalmente una forma ovoidale con dimensioni piuttosto discrete se confrontati con nidi di altre specie.\cite{Schmolz}

Mentre per la specie Vespula vulgaris si osserva che la carta che compone i favi risulta essere più sottile rispetto alla carta dell'involucro. \cite{Cole2001}
Inoltre una proprietà che differisce tra Vespula vulgaris e altre Vespinae (come Dolichovespula sylvestris e Dolichovespula norwegica) è che la densità della carta usata per realizzare i favi è inferiore rispetto a quella della carta usata per l'involucro. \cite{Cole2001}
I favi sono strutturalmente complessi, funzionando come una trave o una mensola che supporta il proprio peso e quello della covata; tuttavia, la loro carta può essere più sottile, rispetto alla struttura esterna, in quanto beneficiano di rinforzi secondari. \cite{Cole2001}
L'involucro esterno, non avendo questi supporti secondari, deve essere più spesso per svolgere efficacemente le sue funzioni di impermeabilità, difesa e isolamento termico. \cite{Cole2001}

Le differenze nelle proprietà fisiche della carta sono correlate al sito di nidificazione tipico di ciascuna specie. \cite{Cole2001} 
I nidi dei Vespinae sono generalmente costruiti all'interno di un involucro in quanto, a differenza di molti generi tropicali, i Vespinae non presentano prove dell'utilizzo di un repellente chimico efficace contro i predatori come le formiche. \cite{Jeanne}
Di conseguenza, l'involucro aiuta a limitare l'accesso alla covata a una stretta apertura, che può essere difesa attivamente dalle vespe "guardiane". \cite{Jeanne}
Vespula vulgaris è un esempio di specie nidificante in cavità che sceglie di nidificare quasi esclusivamente in cavità o in siti sotterranei. \cite{Cole2001}
Data questa scelta di siti protetti, l'involucro del nido è meno esposto agli agenti atmosferici. \cite{Cole2001}
La carta prodotta da V. vulgaris è stata qualificata come fragile e la sua minore resistenza strutturale è in linea con l'habitat protetto in cui nidifica, suggerendo che la velocità di costruzione potrebbe essere più cruciale della qualità del materiale. \cite{Cole2001}
Al contrario, le specie che nidificano in siti esposti o semi-aperti, come le Dolichovespula, necessitano di un involucro più robusto e una maggiore resistenza alla trazione, specialmente nella direzione delle bande di polpa. \cite{Cole2001}
Le colonie di V. vulgaris sono tipicamente grandi. Vespa crabro, sebbene sia la più grande delle Vespinae temperate, costruisce nidi che possono essere ospitati in cavità di alberi o altri luoghi riparati. \cite{Schmolz}

\section{Materiale}
La struttura dei nidi dei Vespidi è strettamente legata alla natura dei materiali impiegati nella loro costruzione, i quali variano in composizione chimica, origine e organizzazione molecolare in funzione delle esigenze ecologiche e comportamentali delle specie. I materiali utilizzati derivano principalmente da fonti vegetali o minerali e vengono rielaborati attraverso secrezioni salivari, che ne modificano le proprietà fisiche e meccaniche, consentendo di ottenere un materiale composito di elevata efficienza strutturale.

Dal punto di vista compositivo, la cellulosa costituisce il principale componente dei nidi cartacei. Essa è ottenuta a partire da fibre vegetali masticate e amalgamate con saliva, formando una matrice fibrosa coesa e resistente. Analisi morfologiche e chimiche, condotte mediante microscopia elettronica a scansione (SEM) e spettroscopia a dispersione di energia (EDX), hanno evidenziato come tali strutture siano costituite prevalentemente da carbonio, ossigeno e azoto, con tracce di elementi inorganici quali silicio, calcio e magnesio, derivanti dal substrato ambientale \cite{Bağriaçik}. Le secrezioni orali, ricche di proteine e amminoacidi, svolgono un ruolo adesivo e protettivo, contribuendo alla coesione delle fibre e conferendo al materiale finale proprietà idrofobiche e resistenza agli agenti atmosferici.

Quando il materiale di nidificazione è di origine minerale, come fango o particolato terroso, la composizione risulta dominata da silicati e carbonati di calcio, con un contenuto organico minore ma con una maggiore resistenza meccanica e stabilità strutturale (Khan et al., 2018). Anche la componente cellulosica, laddove presente, può mostrare diversi gradi di ordine molecolare: forme cristalline (cellulosa alfa e beta) e paracristalline coesistono, conferendo al materiale proprietà variabili di flessibilità, densità e capacità di trattenere umidità \cite{LARSSON199719}.

\section{Materiale fibroso}
Le differenti famiglie dei vespidi realizzano i propri nidi con una ampia varietà di materiali fibrosi \cite{Borges}.
Tra questi materiali alcuni dei più utilizzati sono fibre di origine vegetale quali legno alterato o sano, cellule cuticolari fogliari e peli fogliari \cite{Cole2001}, oltre a segatura, tricomi, epidermide fogliare, alghe e polline \cite{Borges}. 
Per quello che rigurda le specie appartenenti al genere Polistes abbiamo principalmente fibre derivanti dal legno esposto ad agenti atmosferici, ossia fibre vegetali lunghe e peli vegetali.
La scelta di queste fibre influenza poi ovviamente le proprietà del nido realizzato, altri fattori da tenere in considerazione sono anche il sito di nidificazione e sopratutto il diverso tempo di masticazione dei materiali prima di essere incorporati nel nido. \cite{Bagriacik2013}

Esistono anche alcune specie particolari, come Vespula vulgaris oppure come Vespa crabro, che realizzano in modo differente l'involucro esterno del nido e le celle interne dove quindi si hanno differenze nelle proprietà dei materiali fibrosi utilizzati.
Specie come appunto Vespula vulgaris, ma anche altre quali Dolichovespula sylvestris e Dolichovespula norwegica, si può osservare che le fibre presenti nelle celle interne risultano essere significativamente più corte di quelle utilizzate per l'involucro esterno
Tali differenze nella lunghezza delle fibre possono derivare o dalla selezione di diverse fonti di polpa o da una differente intensità del processo di masticazione. 


Ad esempio, Vespula vulgaris utilizza frammenti corti ("short chunks") di materiale legnoso e può raccogliere rapidamente blocchi di legno marcio, risultando in una carta di qualità inferiore ma con un risparmio di tempo nella raccolta rispetto alle specie che utilizzano fibre sane, come le Dolichovespula (che impiegano fibre lunghe e sottili). 
La lunghezza e la composizione delle fibre sono direttamente correlate alla resistenza della carta

\subsection{Cellulosa}
I nidi costruiti dalle specie appartenenti alla famiglia Vespidae, noti per la loro struttura cartacea, si basano fondamentalmente sulla cellulosa come costituente primario della componente fibrosa. 
Questo materiale, tipicamente derivato da fibre vegetali, spesso legno alterato o marcescente, viene processato e aggregato dalle vespe con l'ausilio di una secrezione orale proteica, la quale agisce da legante, rafforzando la struttura e contribuendo alla sua resistenza meccanica. 
La cellulosa, in quanto principale composto organico dell'involucro del nido, è cruciale per le proprietà fisiche del materiale, fornendo un'elevata resistenza alla trazione e un basso peso specifico. 
A livello ultrastutturale e chimico, la cellulosa presente in queste strutture manifesta una complessa architettura molecolare. 
L'analisi mediante spettroscopia di risonanza magnetica nucleare del carbonio ( 13C NMR) con polarizzazione incrociata e rotazione all'angolo magico (CP/MAS  13C NMR) ha permesso di elucidare il polimorfismo strutturale di questo biopolimero. 
I domini cristallini della cellulosa nativa, che costituisce la base della fibra vegetale impiegata, sono composti da una miscela di due allomorfi distinti: la cellulosa Ialfa e la cellulosa Ibeta. 
Sebbene la cellulosa Ibeta sia la forma predominante nelle piante superiori, studi approfonditi hanno rivelato la presenza diffusa di segnali spettrali aggiuntivi, non direttamente riconducibili agli allomorfi cristallini Ialfa e Ibeta. 
Nello specifico, la quantificazione spettrale ha identificato un segnale caratteristico nell'intervallo C-4 ( 88.1-88.5) e C-1 ( 104.5). 
Questa componente, rilevata in diverse fonti di cellulosa nativa, inclusa la cellulosa di legno, è stata attribuita a una struttura 'in core' meno ordinata o para-cristallina. 
Tale isoforma manifesta una mobilità molecolare superiore rispetto agli allomorfi cristallini Iaòfa e Ibeta, suggerendo che l'architettura fibrosa dei nidi delle Vespidi incorpora stati di ordine della cellulosa più eterogenei e complessi rispetto alla semplice dicotomia tra domini amorfi e cristallini convenzionali.

\section{Materiali proteici}
I materiali proteici presenti nei nidi dei vespidi sociali derivano prevalentemente da una secrezione orale (saliva), che funge da legante o adesivo per le fibre vegetali \cite{Schmolz}. 
Questa secrezione è descritta come una proteina simil-seta (silklike protein) o mucoproteina, e viene aggiunta alla polpa di cellulosa durante il processo di costruzione \cite{Singer1992,Espelie1990}. 
Chimicamente, questa secrezione si indurisce rapidamente e irreversibilmente in una sostanza cornea insolubile e idrorepellente, essenziale per l'impermeabilizzazione e il rafforzamento del nido \cite{Schmolz,Curtis}. 
L'analisi degli amminoacidi della proteina di nido in Polistes metricus ha mostrato un'alta concentrazione di glicina, serina, alanina e prolina (che insieme costituiscono il 65-73 percento dei residui identificati) \cite{Singer1992}. 
In particolare, la presenza relativamente elevata di prolina (10.6-12.5 percento nei nidi di P. metricus) è significativa, in quanto contribuisce alla resistenza strutturale del nido, essendo un amminoacido dominante nelle proteine strutturali \cite{Singer1992}.
Nel contesto specifico di Polistes dominula, l'allocazione di materiale proteico alla costruzione del nido è stata studiata come un compromesso (trade-off) tra la robustezza del nido e la produttività della prole.
Nello specifico, in condizioni di foraggiamento naturale (scarsità di prede), P. dominula utilizzava una concentrazione proteica significativamente inferiore nel materiale del nido rispetto a Polistes fuscatus. 
Questo suggerisce che P. dominula (specie invasiva) può allocare meno proteine per la secrezione orale, destinandone di più alla prole in via di sviluppo, il che contribuisce a una maggiore produttività in condizioni di risorse limitate. 
Le specie di Polistes come P. gallicus e P. dominula, oltre alla secrezione orale per l'incollaggio della carta, utilizzano anche una secrezione ghiandolare sternale sul peduncolo del nido come deterrente chimico contro i predatori, in particolare le formiche. 
Questa secrezione repulsiva è composta principalmente da lipidi e acidi grassi liberi (come l'acido esadecanoico e l'acido ottadecenoico, rilevati nel peduncolo di P. annularis e nelle ghiandole sternali di P. fuscatus e P. annularis), che pur non essendo proteine, sono cruciali per la difesa chimica.

Studi condotti sui nidi della specie Polistes dominula hanno permesso di individuare un gruppo di proteine con la probabile funzione di impermeabilizzare il nido \cite{StanfordBrownSpelman2014_MaterialWaterproofing}.
Partendo poi da questo studio si è andato a ricercare la sequenza aminoacidica relativa, dalla quale è stata predetta la struttura tridimensionale usando Alphafold \cite{AlphaFold}.

\begin{figure}[H]
  \centering
  \includegraphics[width=1\textwidth]{Figures/IMG03231.png}
  \caption{Struttura della proteina PdomMRNAr1.2-03231.1 predetta con Alphafold}
  \label{fig:esempio-immagine}
\end{figure}


\begin{figure}[H]
  \centering
  \includegraphics[width=1\textwidth]{Figures/IMG08705.png}
  \caption{Struttura della proteina PdomMRNAr1.2-08705.1 predetta con Alphafold}
  \label{fig:esempio-immagine}
\end{figure}

\begin{figure}[H]
  \centering
  \includegraphics[width=0.5\textwidth]{Figures/IMG10508.png}
  \caption{Struttura della proteina PdomMRNAr1.2-10508.1 predetta con Alphafold}
  \label{fig:esempio-immagine}
\end{figure}

\begin{figure}[H]
  \centering
  \includegraphics[width=0.5\textwidth]{Figures/IMG04156.png}
  \caption{Struttura della proteina PdomMRNAr1.2-04156.1 predetta con Alphafold}
  \label{fig:esempio-immagine}
\end{figure}


Sono state riportate le strutture di quattro delle proteine:
\begin{itemize}
    \item PdomMRNAr1.2-03231.1
    \item PdomMRNAr1.2-08705.1 
    \item PdomMRNAr1.2-10508.1 
    \item PdomMRNAr1.2-04156.1
\end{itemize}

Le strutture di queste proteine sono state predette con successo, ciò è dimostrato dalle sequenze di colore blu e azzuro nelle immagini precedenti. 
L'unica incertezza, mostrata nella predizione con un colore arancione della catena, è associata ad una porzione della proteina PdomMRNAr1.2-04156.1.

Altre due proteine erano stato individuate, tuttavia per una di queste non era stato possibile determinare la sequenza aminoacidica, mentre per l'altra non si è riusciti ad ottenere una predizione della struttura soddisfacente.
