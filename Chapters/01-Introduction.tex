% !TEX root = ../Thesis.tex

\chapter{Introduzione}
\section{La famiglia Vespidae}
La famiglia dei Vespidae rappresenta un gruppo estremamente diversificato di Imenotteri, comprendente più di 5000 specie differenti distribuite tra sei sottofamiglie principali: Stenogastrinae, Eupragiinae, Masarinae, Eumeninae, Polistinae e Vespinae \cite{Khan}.


Tra queste quelle che possono destare più interesse sono specie dei generi Polistes e Vespula, appartenenti rispettivamente alle sottofamiglie Polistinae e Vespinae; queste vespe hanno la caratteristica di essere specie eusociali \cite{Khan,Goulet}, ossia vi è una netta idvisisone in caste tra regina, operai e maschi.


Le specie appartenenti a questi generi si prestano in modo ottimale allo studio di diverse proprietà e comportamneti caratteristici, in particolare in questo studio è di interesse il diverso modo di nidificare, i differenti materiali usati durante la nidifiaczione e la relazione tra i materiali utilizzati e l'ambiente circostante.\\ 

\section{Architettura nidi dei vespidi}
Lo studio delle diverse proprietà di nidificazione può essre interessante sotto diversi aspetti, in quanto comprendere le diverse condizioni con cui le vespe costruiscono i propri nidi permette anche di analizzare il loro ruolo ecologico zzz [3].


Ciascuna specie presenta può presentare delle piccole differenze dalle altre a livello di struttura e morfologia dei nidi, dovuti a diversi processi di adattabilità ambientale. Tuttavia è possibile ricondurre a cartteristiche comuni, per esmepio vespe sociali tendono a costruire nidi con una archittetura più complessa zzz.



\section{Polistes}
Andando ad osservare nel dettaglio le abitudini di vespe del genere Polistes (in particolare Polistes Dominula e Polistes Gallicus) vedremo che questi vespidi realizzano i propri nidi con un materiale di tipo cartaceo, che viene prodotto tramite l'utilizzo di fibre vegetali, per esempio frammenti di corteccia, che vengono masticati e quindi mescolati con secrezioni salivari di natura proteica; queste hanno lo scopo li legare le fibre vegetali e formare un nido compatto, am anche quello di impermeabilizzare il nido. A livello morfologico queste specie danno origine a nidi zzz  Le Polistes (es. P. dominula, P. gallicus) costruiscono nidi aerei aperti, costituiti da un solo favo di celle esagonali sospeso tramite un peduncolo (petiolo) [8]. Questi nidi, privi di involucro esterno, sono realizzati con un materiale di tipo “cartaceo” ottenuto dalla masticazione di fibre vegetali (legno, corteccia o steli secchi) mescolate a secrezioni salivari proteiche che fungono da agente legante e impermeabilizzante [9,10].
La secrezione orale è vitale per la protezione del nido dagli agenti atmosferici. La quantità di saliva è specie-specifica, come dimostrato dalla notevole variazione tra P. nimpha (58) e le altre due specie (22-23) \cite{Bağriaçik}


\section{Vespa crabro}
Le Vespula e le Vespa (es. Vespula vulgaris, Vespa crabro, Vespa velutina) costruiscono invece nidi più complessi, generalmente chiusi e multi-strato, composti da diversi favi sovrapposti e protetti da un involucro esterno [11]. Tali nidi possono essere aerei o sotterranei, a seconda della specie, e sono costituiti da una carta più spessa e compatta, con fibre vegetali più corte e una maggiore quantità di materiale inorganico, come polveri e particelle di suolo [12]


