% !TEX root = ../Thesis.tex


\chapter{Introduzione}

Le vespe appartenenti alla famiglia Vespidae rappresentano un gruppo di insetti sociali di notevole interesse sia ecologico che chimico-biologico, grazie alla complessità dei materiali che impiegano nella costruzione dei propri nidi \cite{Cole2001,Jeanne}.
Questi materiali, derivati principalmente da fonti vegetali, come le fibre di legno, e da secrezioni di origine animale, come la saliva (secrezioni orali proteiche), costituiscono biocompositi naturali nei quali componenti cellulosiche e proteiche interagiscono conferendo al nido specifiche proprietà meccaniche, strutturali e di resistenza ambientale, inclusa l'impermeabilizzazione \cite{Curtis,Singer1992}.
Lo studio della composizione chimica e strutturale dei nidi consente di comprendere non solo le strategie costruttive e adattative delle diverse specie di vespe, ma anche di individuare analogie con materiali polimerici naturali di potenziale interesse per la scienza dei materiali, fungendo da ispirazione per nuove soluzioni tecnologiche.

In questa tesi si analizzano nidi di diverse specie appartenenti alla famiglia Vespidae, con particolare attenzione alla componente cellulosica e proteica, mediante tecniche di spettroscopia NMR allo stato solido e altre analisi complementari. L’obiettivo è quello di correlare la composizione chimica dei nidi con le differenti modalità costruttive e ambientali delle specie considerate, contribuendo alla comprensione dei processi biochimici e strutturali alla base della formazione di questi peculiari biocompositi naturali.

\section{La famiglia Vespidae}

La famiglia Vespidae rappresenta un gruppo estremamente diversificato di Imenotteri, la diversità tassonomica dei Vespidi può essere rappresentata attraverso la loro filogenesi, che mostra la suddivisione in differenti sottofamiglie e generi quali Polistes, Vespula e Vespa. 
L’albero riportato in Figura 1 evidenzia le relazioni tra alcune delle specie considerate nello studio. \cite{Perrard}
Questa diversità è evidenziata con una vasta gamma di abitudini sociali e morfologiche, si hanno infatti vespidi solitari che hanno abitudine nel realizzare nidi in fango, mentre invece si osserva che vespe di specie eusociali sfruttano la carta come componente principali dei nidi.\cite{Jeanne}
Sono per l'appunto delle vespe sociali le quattro specie di vespidi che sono stati studiati, ossia Polistes dominula, Polistes galliccus, Vespula vulgaris e Vespa crabro.

\begin{figure}[H]
  \centering
  \includegraphics[width=1\textwidth]{ALBVESP2.png}
  \caption{Albero filogenetico dei Vespidi, con la posizione dei generi Polistes, Vespula e Vespa.}
  \label{fig: albero filogenetico}
\end{figure}

L'architettura del nido è un tratto evolutivo cruciale, plasmato da fattori ambientali e dalla forte pressione predatoria, in particolare da parte delle formiche, nelle regioni tropicali. \cite{Jeanne}
Si possono distinguere due principali strategie architettoniche, specie del genere Polistes per esempio realizzano nidi con favo scoperto sospesi tramite un sottile peduncolo difeso chimicamente con una secrezione ghiandolare repellente, questa tipologie di nidi sono detti stelocittari gimnodomi. \cite{Jeanne}
L'altra tipologia è rappresentata dai nidi caliptomi tipico dei generi Vespula e Vespa, sono nidi caratterizzati da un favo racchiuso in un involucro dove l'accesso è rappresentato da un'unica fessura o foro, nidi di questo tipo facilitando la difesa e forniscono anche isolamento termico. \cite{Jeanne,Cole2001,Schmolz}
Il materiale primario di costruzione è il cartone, che si origina dalla miscelazione di fibre vegetali (cellulose), talvolta legno alterato o triconi specifici, che ne influenzano la resistenza e la densità, con una secrezione orale proteica ricca di aminoacidi come prolina, glicina e serina, che funge da matrice adesiva, conferendo robustezza e idrorepellenza. \cite{Jeanne}


\section{Materiali costitutivi dei nidi}
La struttura dei nidi dei Vespidi è strettamente legata alla natura dei materiali impiegati nella loro costruzione, i quali variano in composizione chimica, origine e organizzazione molecolare in funzione delle esigenze ecologiche e comportamentali delle specie. \cite{Hozumi,Jeanne} 
I materiali utilizzati derivano principalmente da fonti vegetali o minerali e vengono rielaborati attraverso secrezioni salivari, che ne modificano le proprietà fisiche e meccaniche, consentendo di ottenere un materiale composito di elevata efficienza strutturale.

Dal punto di vista compositivo, la cellulosa costituisce il principale componente dei nidi cartacei, essa è ottenuta a partire da fibre vegetali masticate e amalgamate con saliva, formando una matrice fibrosa coesa e resistente. \cite{Singer1992}
Analisi morfologiche e chimiche, condotte mediante microscopia elettronica a scansione (SEM) e spettroscopia a dispersione di energia (EDX), hanno evidenziato come tali strutture siano costituite prevalentemente da carbonio, ossigeno e azoto, con tracce di elementi inorganici quali silicio, calcio e magnesio, derivanti dal substrato ambientale \cite{Bağriaçik}. 
Le secrezioni orali, ricche di proteine e amminoacidi, svolgono un ruolo adesivo e protettivo, contribuendo alla coesione delle fibre e conferendo al materiale finale proprietà idrofobiche e resistenza agli agenti atmosferici. \cite{Curtis,Schmolz}

Quando il materiale di nidificazione è di origine minerale, come fango o particolato terroso, la composizione risulta dominata da silicati e carbonati di calcio, con un contenuto organico minore ma con una maggiore resistenza meccanica e stabilità strutturale \cite{Khan}. 
Anche la componente cellulosica, laddove presente, può mostrare diversi gradi di ordine molecolare: forme cristalline (cellulosa alfa e beta) e paracristalline coesistono, conferendo al materiale proprietà variabili di flessibilità, densità e capacità di trattenere umidità \cite{LARSSON199719}.

\subsection{Materiale fibroso}
Le differenti famiglie dei vespidi realizzano i propri nidi con una ampia varietà di materiali fibrosi \cite{Borges}.
Tra questi materiali alcuni dei più utilizzati sono fibre di origine vegetale quali legno alterato o sano, cellule cuticolari fogliari e peli fogliari \cite{Cole2001}, oltre a segatura, tricomi, epidermide fogliare, alghe e polline \cite{Borges}. 
Per quello che rigurda le specie appartenenti al genere Polistes abbiamo principalmente fibre derivanti dal legno esposto ad agenti atmosferici, ossia fibre vegetali lunghe e peli vegetali. \cite{Bagriacik2013}
La scelta di queste fibre influenza poi ovviamente le proprietà del nido realizzato, altri fattori da tenere in considerazione sono anche il sito di nidificazione e sopratutto il diverso tempo di masticazione dei materiali prima di essere incorporati nel nido. \cite{Bagriacik2013}

Si hanno poi specie come Vespula vulgaris oppure come Vespa crabro, che realizzano in modo differente l'involucro esterno del nido e le celle interne, dove quindi si hanno differenze nelle proprietà dei materiali fibrosi utilizzati. \cite{Cole2001,Schmolz}
In specie come appunto Vespula vulgaris, ma anche altre quali Dolichovespula sylvestris e Dolichovespula norwegica, si può osservare che le fibre presenti nelle celle interne risultano essere significativamente più corte di quelle utilizzate per l'involucro esterno.
Tali differenze nella lunghezza delle fibre possono derivare o dalla selezione di diverse fonti di polpa o da una differente intensità del processo di masticazione. \cite{Cole2001}

\subsection{Cellulosa}

I nidi costruiti da questi vespidi, noti per la loro struttura cartacea, si basano fondamentalmente sulla cellulosa come costituente primario della componente fibrosa. \cite{Curtis}
Questo materiale viene processato e aggregato dalle vespe con l'ausilio di una secrezione orale proteica.\cite{Singer1992}
La cellulosa, in quanto principale composto organico dell'involucro del nido, è cruciale per le proprietà fisiche del materiale, fornendo un'elevata resistenza alla trazione e un basso peso specifico. \cite{Schmolz}

A livello ultrastutturale e chimico, la cellulosa presente in queste strutture manifesta una complessa architettura molecolare. 
L'analisi mediante spettroscopia di risonanza magnetica nucleare del carbonio ha permesso di elucidare il polimorfismo strutturale di questo biopolimero. 
I domini cristallini della cellulosa nativa, che costituisce la base della fibra vegetale impiegata, sono composti da una miscela di due allomorfi distinti: cellulosa alfa e cellulosa beta. 
Sebbene la cellulosa beta sia la forma predominante nelle piante superiori, studi approfonditi hanno rivelato la presenza diffusa di segnali spettrali aggiuntivi, non direttamente riconducibili agli allomorfi cristallini alfa e beta. 
Nello specifico, la quantificazione spettrale ha identificato un segnale caratteristico che è stato attribuito a una struttura meno ordinata o para-cristallina. 
Tale isoforma manifesta una mobilità molecolare superiore rispetto agli allomorfi cristallini, suggerendo che l'architettura fibrosa dei nidi delle Vespidi incorpora stati di ordine della cellulosa più eterogenei e complessi rispetto alla semplice dicotomia tra domini amorfi e cristallini convenzionali. \cite{LARSSON199719}

\subsection{Materiale proteico}
I materiali proteici presenti nei nidi dei vespidi sociali derivano prevalentemente da secrezione orale, che funge da legante o adesivo per le fibre vegetali \cite{Schmolz}. 
Questa secrezione è descritta come una proteina simil-seta o mucoproteina, e viene aggiunta alla polpa di cellulosa durante il processo di costruzione \cite{Singer1992,Espelie1990}. 
Chimicamente, questa secrezione si indurisce rapidamente e irreversibilmente in una sostanza cornea insolubile e idrorepellente, essenziale per l'impermeabilizzazione e il rafforzamento del nido \cite{Schmolz,Curtis}. 

Le specie del genere Polistes incorporano una secrezione orale proteica, definita silklike protein, per cementare le fibre cartacee \cite{Singer1992}.
Tale proteina strutturale è caratterizzata da un elevato contenuto di amminoacidi quali prolina, glicina, alanina e serina, che conferiscono robustezza al materiale \cite{Singer1992,Curtis}. 
Un altro elemento caratteristico delle specie del genere Polistes è l'applicazione di una secrezione ghiandolare repellente sul peduncolo per la difesa chimica contro le formiche \cite{Jeanne}. 
Al contrario, Vespula e Vespa, i quali costruiscono nidi chiusi da un involucro, utilizzano materiale proteico descritto come una secrezione simile alla chitina come agente legante per la carta; non è stato riscontrato invece l'uso di repellenti chimici attivi come in Polistes \cite{Jeanne}.

Studi condotti sui nidi della specie Polistes dominula hanno permesso di individuare un gruppo di proteine con la probabile funzione di impermeabilizzare il nido \cite{StanfordBrownSpelman2014_MaterialWaterproofing}.
Partendo poi da questo studio si è andato a ricercare la sequenza aminoacidica relativa, dalla quale è stata predetta la struttura tridimensionale usando Alphafold \cite{AlphaFold}.

\begin{figure}[H]
  \centering
  \includegraphics[width=1\textwidth]{Figures/IMG03231.png}
  \caption{Struttura della proteina PdomMRNAr1.2-03231.1 predetta con Alphafold}
  \label{fig:prot1}
\end{figure}


\begin{figure}[H]
  \centering
  \includegraphics[width=1\textwidth]{Figures/IMG08705.png}
  \caption{Struttura della proteina PdomMRNAr1.2-08705.1 predetta con Alphafold}
  \label{fig:prot2}
\end{figure}

\begin{figure}[H]
  \centering
  \includegraphics[width=0.5\textwidth]{Figures/IMG10508.png}
  \caption{Struttura della proteina PdomMRNAr1.2-10508.1 predetta con Alphafold}
  \label{fig:prot 3}
\end{figure}

\begin{figure}[H]
  \centering
  \includegraphics[width=0.5\textwidth]{Figures/IMG04156.png}
  \caption{Struttura della proteina PdomMRNAr1.2-04156.1 predetta con Alphafold}
  \label{fig:prot 4}
\end{figure}


Sono state riportate le strutture di quattro delle proteine:
\begin{itemize}
    \item PdomMRNAr1.2-03231.1
    \item PdomMRNAr1.2-08705.1 
    \item PdomMRNAr1.2-10508.1 
    \item PdomMRNAr1.2-04156.1
\end{itemize}

Le strutture di queste proteine sono state predette con successo, ciò è dimostrato dalle sequenze di colore blu e azzuro nelle immagini precedenti. 
L'unica incertezza, mostrata nella predizione con un colore arancione della catena, è associata ad una porzione della proteina PdomMRNAr1.2-04156.1.

Altre due proteine erano stato individuate, tuttavia per una di queste non era stato possibile determinare la sequenza aminoacidica, mentre per l'altra non si è riusciti ad ottenere una predizione della struttura soddisfacente.
