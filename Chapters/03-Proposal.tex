% !TEX root = ../Thesis.tex

% ...existing code...
\chapter{Metodi di analisi}
\label{cap:proposal}

Lo scopo principale delle attività sperimentali è stato quello di determinare la composizione e struttura dei nidi di diverse specie appartenenti alla famiglia vespidae.
Sono stati analizzati campioni provenienti da nidi di Polistes dominula, Polistes gallicus, Vespula spp. e Vespa crabro, di questi ultimi si è analizzato sia l'involucro esterno sia le celle interne dei nidi.

\noindent Per preservare le caratteristiche dei campioni dei nidi, e quindi evitare trattamenti preliminare di qualsiasi natura, si è proceduto all'analisi di questi tramite la tecnica di spettroscopia NMR a stato solido, che permette di ricavare importanti informazioni riguardo la struttura e l'organizzazione delle componeneti dei nidi.

\noindent Le analisi sono state condotte due diversi spettrometri che lavorano con intensità di campo differente, 700 e 800; nel presente capitolo si descrive la procedura di preparazione del campione, le diverse analisi effettuate sui campioni e la raccolta dei dati sperimentali.

\section{Preparazione dei campioni}
I campioni dei nidi che sono stati utilizzati sono:
\begin{itemize}
    \item Polistes dominula
    \item Polistes gallicus
    \item Vespa crabro - celle interne e involucro esterno
    \item Vespula - celle interne e involucro esterno
\end{itemize}
Il vantaggio della tecnica utilizzata è quello di poter usare il campione desiderato tale e quale allo stato solido, ossia senza bisogno di trattamenti chimici o di trovare un solvente adeguato per preparare una soluzione da analizzare.
L'unica necessità preparativa è quella di ottenere il campione finemente macinato, quindi i diversi campioni sono stati tutti triturati con un pestello in un mortaio fino a raggiungere una polvere fine e omogenenea; è infatti importante per la qualità dei risultati che le dimensioni dei granuli che si ottengono siano il più possibile simili.

\noindent Dopo questa breve preparazione i campioni sono stati introdotti con un apposito imbuto in un rotore di 3.2 mm, questo è stato poi compattato il più possibile in modo ceh all'interno ci fosse più campione possibile.
Una volta preparato correttamente il rotore questo è stato chiuso ed è stato fatto un segno con un pennarello in modo che lo spettrometro tramite un sensore IR rilevasse la velocità di rotazione del rotore durante l'analisi.

\section{Tecniche di analisi}
Per andare a studiare la struttura molecolare dei nidi si è sfruttata la tecnica di spettroscopia NMR allo stato solido del carbonio-13 (¹³C CP/MAS NMR), che consente di ricavare informazioni sulla struttura tramite il segnale associato al ¹³C.
Essa combina la polarizzazione incrociata (Cross Polarization, CP), che aumenta la sensibilità del segnale ¹³C trasferendo magnetizzazione dai nuclei di ¹H, con la rotazione ad angolo magico (Magic Angle Spinning, MAS), che riduce l’anisotropia delle interazioni dipolari e dello spostamento chimico.
Questa configurazione permette di ottenere spettri ad alta risoluzione, utili per distinguere le diverse tipologie di carbonio presenti e valutare il grado di ordine o disordine strutturale del materiale analizzato, in questo caso i nidi delle specie della famiglia vespidae

\begin{figure}[htbp]
  \centering
  \includegraphics[width=0.7\textwidth]{Figures/CPMAS13C.png}
  \caption{Schematica rappresentazione del funziomento dello strumento}
  \label{fig:esempio-immagine}
\end{figure}

Per i diversi campioni sono state effetuate misure con due differenti spettrometri che lavorano ad una diversa intensità del campo magnetico, ossia 700 MHZ ( che corripsonde ad un campo magnetico di 16.4 T) e 800 MHz ( che corripsonde ad un campo magnetico di 18.8 T).
In particolare sono stati analizzati con lo strumento a 700 MHz i campioni di:
\begin{itemize}
    \item Polistes dominula 
    \item Polistes gallicus 
    \item Vespa crabro - celle interne e involucro esterno
\end{itemize}
Mentre con lo strumento a 800 MHz sono stati analizzati i campioni di:
\begin{itemize}
    \item Polistes dominula
    \item Vespa crabro - celle interne e involucro esterno
    \item Vespula - celle interne e involucro esterno
\end{itemize}
Le analisi effettuate con lo spettrometro a 800 MHz consente di ottenere una maggiore sensibilità e una migliore risoluzione spettrale. Il campione di Vespa crabro è stato analizzato con entrambi gli strumenti in quanto gli spettri ottenuti con lo spettrometro a 700 MHz sono stati esclusi dall’analisi a causa di problemi tecnici riscontrati nel funzionamento del probe durante la misura.
Invece per i nidi di Polistes dominula sono stati registrati gli spettri di entrambi gli spettrometri poichè si voleva un confronto tra i due spettri ottenuti e sopratutto un confronto riguardante i chemical shift dei carboni in posizione 1 e posizione 4 presenti in letteratura \cite{LARSSON199719} che sono stati determinati proprio con uno strumento a 700 MHz

\section{Risulati e spettri ottenuti}

\begin{figure}[H]
  \centering
  \includegraphics[width=1\textwidth]{1D_Spectrum_Dominula_vecchio_899.png}
  \caption{Spettro ¹³C CP/MAS NMR del campione più vecchio di Polistes dominula (700MHz)}
  \label{fig:Dominula vecchio 700}
\end{figure}

\begin{figure}[H]
  \centering
  \includegraphics[width=1\textwidth]{1D_Spectrum_Dominula_nuovo_899.png}
  \caption{Spettro ¹³C CP/MAS NMR del campione più vecchio di Polistes dominula (800MHz)}
  \label{fig:Dominula vecchio 800}
\end{figure}

\begin{figure}[H]
  \centering
  \includegraphics[width=1\textwidth]{1D_Spectrum_Dominula_nuovo_5.png}
  \caption{Spettro ¹³C CP/MAS NMR del campione più nuovo di Polistes dominula (800MHz)}
  \label{fig:Dominula nuovo 800}
\end{figure}

\begin{figure}[H]
  \centering
  \includegraphics[width=1\textwidth]{1D_Spectrum_Gallicus_nuovo_899.png}
  \caption{Spettro ¹³C CP/MAS NMR del campione di Polistes gallicus (700MHz)}
  \label{fig:Gallicus 700}
\end{figure}

\begin{figure}[H]
  \centering
  \includegraphics[width=1\textwidth]{1D_Spectrum_Crabro_25.png}
  \caption{Spettro ¹³C CP/MAS NMR delle celle di Vespa crabro (800MHz)}
  \label{fig:celle vespa crabro}
\end{figure}

\begin{figure}[H]
  \centering
  \includegraphics[width=1\textwidth]{1D_Spectrum_Crabro_898.png}
  \caption{Spettro ¹³C CP/MAS NMR dell'involucro di Vespa crabro (800MHz)}
  \label{fig:inv vespa crabro}
\end{figure}

\begin{figure}[H]
  \centering
  \includegraphics[width=1\textwidth]{1D_Spectrum_Vespula_5.png}
  \caption{Spettro ¹³C CP/MAS NMR delle celle di Vespula vulgaris (800MHz)}
  \label{fig:Dominula vecchio 700}
\end{figure}

\begin{figure}[H]
  \centering
  \includegraphics[width=1\textwidth]{1D_Spectrum_vespula_15.png}
  \caption{Spettro ¹³C CP/MAS NMR dell'involucro di Vespula vulgaris (800MHz)}
  \label{fig:Dominula vecchio 700}
\end{figure}







