% !TEX root = ../Thesis.tex

% ...existing code...
\chapter{Metodi di analisi}
\label{cap:proposal}

Lo scopo principale delle attività sperimentali è stato quello di determinare la composizione e struttura dei nidi di diverse specie appartenenti alla famiglia vespidae.
Sono stati analizzati campioni provenienti da nidi di Polistes dominula, Polistes gallicus, Vespula vulgaris e Vespa crabro, di questi ultimi si è analizzato sia l'involucro esterno sia le celle interne dei nidi.

Per preservare le caratteristiche dei campioni dei nidi, e quindi evitare trattamenti preliminare di qualsiasi natura, si è proceduto all'analisi di questi tramite la tecnica di spettroscopia NMR a stato solido del carbonio (¹³C CP/MAS NMR), che permette di ricavare importanti informazioni riguardo la struttura e l'organizzazione delle componeneti dei nidi.
Le analisi sono state condotte due diversi spettrometri che lavorano con intensità di campo differente, 700 MHz e 800 MHz; nel presente capitolo si descrive la procedura di preparazione del campione, le diverse analisi effettuate sui campioni e la raccolta dei dati sperimentali.

\section{Preparazione dei campioni}
I campioni dei nidi che sono stati utilizzati sono:
\begin{itemize}
    \item Polistes dominula
    \item Polistes gallicus
    \item Vespa crabro - celle interne e involucro esterno
    \item Vespula - celle interne e involucro esterno
\end{itemize}
Il vantaggio della tecnica utilizzata è quello di poter usare il campione desiderato tale e quale allo stato solido, ossia senza bisogno di trattamenti chimici o di trovare un solvente adeguato per preparare una soluzione da analizzare.
L'unica necessità preparativa è quella di ottenere il campione finemente macinato, quindi i diversi campioni sono stati tutti triturati con un pestello in un mortaio fino a raggiungere una polvere fine e omogenenea; è infatti importante per la qualità dei risultati che le dimensioni dei granuli che si ottengono siano il più possibile simili.

Dopo questa breve preparazione i campioni sono stati introdotti con un apposito imbuto in un rotore di 3.2 mm, questo è stato poi compattato il più possibile in modo ceh all'interno ci fosse più campione possibile.
Una volta preparato correttamente il rotore questo è stato chiuso ed è stato fatto un segno con un pennarello in modo che lo spettrometro tramite un sensore IR rilevasse la velocità di rotazione del rotore durante l'analisi.

\section{Tecniche di analisi}
Per andare a studiare la struttura molecolare dei nidi si è sfruttata la tecnica di spettroscopia NMR allo stato solido del carbonio-13 (¹³C CP/MAS NMR), che consente di ricavare informazioni sulla struttura tramite il segnale associato al ¹³C.
Essa combina la polarizzazione incrociata (Cross Polarization, CP), che aumenta la sensibilità del segnale ¹³C trasferendo magnetizzazione dai nuclei di ¹H, con la rotazione ad angolo magico (Magic Angle Spinning, MAS), che riduce l’anisotropia delle interazioni dipolari e dello spostamento chimico.
Questa configurazione permette di ottenere spettri ad alta risoluzione, utili per distinguere le diverse tipologie di carbonio presenti e valutare il grado di ordine o disordine strutturale del materiale analizzato, in questo caso i nidi delle specie della famiglia vespidae

\begin{figure}[htbp]
  \centering
  \includegraphics[width=0.7\textwidth]{Figures/CPMAS13C.png}
  \caption{Schematica rappresentazione del funziomento dello strumento}
  \label{fig:CPMAS13C}
\end{figure}

Gli esperimenti di spettroscopia NMR allo stato solido sono stati registrati su due spettrometri Bruker Avance II, uno operante a una frequenza di Larmor di ¹H pari a 700 MHz (16,4 T), corrispondente a una frequenza di Larmor di ¹³C pari a 176 MHz, e l'altro a una frequenza di Larmor di ¹H pari a 700 MHz (18,8 T) corrispondente a una frequenza di Larmor di ¹³C pari a 201 MHz. 
Gli spettrometri erano equipaggiati con una sonda MAS BVT da 3,2 mm (800 MHz) e da 4 mm (700 MHz) in modalità a doppia risonanza. 
La frequenza di rotazione all’angolo magico (MAS) del campione è stata impostata a 11.111 kHz.

Per gli esperimenti monodimensionali {¹H}-¹³C, il tempo di contatto della cross-polarizzazione è stato variato da 200 µs a 2000 µs; gli spettri sono stati coprocessati utilizzando MCR.
Per gli esperimenti bidimensionali {¹H}-¹³C HETCOR, la cross-polarizzazione è stata ottenuta soddisfacendo la condizione di Hartmann–Hahn per k = 1. 
Le finestre spettrali per ¹H e ¹³C erano pari rispettivamente a 20 e 250 ppm. 
il tempo di contatto della cross-polarizzazione è stato variato da 200 µs a 2000 µs.
Durante l’evoluzione della magnetizzazione di ¹H sotto lo spostamento chimico nella dimensione indiretta, è stata utilizzata la sequenza di decoupling PMLG per sopprimere gli accoppiamenti dipolari ¹H–¹H. 
In questi esperimenti, il tempo di rilassamento tra una scansione e la successiva (interscan delay) è stato impostato a 1 s.


In particolare sono stati analizzati con lo spettrometro a 700 MHz:
\begin{itemize}
    \item Entrambi i campioni di Polistes dominula 
    \item Entrambi i campioni di Polistes gallicus 
\end{itemize}
Mentre con lo spettrometro a 800 MHz sono stati analizzati:
\begin{itemize}
    \item Il campione più recente Polistes dominula
    \item Vespa crabro - celle interne e involucro esterno
    \item Vespula - celle interne e involucro esterno
\end{itemize}
Per i nidi di Polistes dominula sono stati registrati gli spettri di entrambi gli spettrometri poichè si voleva un confronto tra i due spettri ottenuti e sopratutto un confronto riguardante i chemical shift dei carboni della cellulosa presenti in letteratura \cite{LARSSON199719}, che sono stati determinati proprio con uno strumento a 700 MHz

\section{Risulati e spettri ottenuti}
Nelle figure sottostanti sono riportati gli spettri monodimensionali dei diversi campioni registrati con un tempo di contatto di 200 µs e 700 µs.
Mentre invece sono stati riportati gli spettri bidimensionali solo dei campioni di Polistes dominula, Polistes gallicus e l'involucro di Vespula vulgaris. Questi spettri riportati corrispondono ad un tempo di contatto di 2000 µs.

\subsection{Spettri monodimensionali}

Negli spettri monodimensionali riportati di seguito sono evidenziate tre differenti zone dello spettro.
La zona in rosso fa riferimento al segnale del gruppo carbonilico delle proteine, altro segnale proteico è messo in evidenza nella zona evidenziata in giallo, infine in blu è mostrata la zona che fa riferimento al segnale dei carboni della cellulosa.

\begin{figure}[H]
  \centering
  \includegraphics[width=1\textwidth]{DomV1D.png}
  \caption{Spettro ¹³C CP/MAS NMR del campione più vecchio di Polistes dominula (700MHz)}
  \label{fig:Dominula vecchio 700}
\end{figure}

\begin{figure}[H]
  \centering
  \includegraphics[width=1\textwidth]{DomN1D.png}
  \caption{Spettro ¹³C CP/MAS NMR del campione più nuovo di Polistes dominula (800MHz)}
  \label{fig:Dominula nuovo 700}
\end{figure}

\begin{figure}[H]
  \centering
  \includegraphics[width=1\textwidth]{DomN8001D.png}
  \caption{Spettro ¹³C CP/MAS NMR del campione più nuovo di Polistes dominula (800MHz)}
  \label{fig:Dominula nuovo 800}
\end{figure}

\begin{figure}[H]
  \centering
  \includegraphics[width=1\textwidth]{GalV1D.png}
  \caption{Spettro ¹³C CP/MAS NMR del campione più vecchio Polistes gallicus (700MHz)}
  \label{fig:Gallicus 700}
\end{figure}

\begin{figure}[H]
  \centering
  \includegraphics[width=1\textwidth]{GalN1D.png}
  \caption{Spettro ¹³C CP/MAS NMR del campione più nuovo di Polistes gallicus (700MHz)}
  \label{fig:Gallicus nuovo 700}
\end{figure}

\begin{figure}[H]
  \centering
  \includegraphics[width=1\textwidth]{CrabroE1D.png}
  \caption{Spettro ¹³C CP/MAS NMR delle celle di Vespa crabro (800MHz)}
  \label{fig:celle vespa crabro}
\end{figure}

\begin{figure}[H]
  \centering
  \includegraphics[width=1\textwidth]{CrabroI1D.png}
  \caption{Spettro ¹³C CP/MAS NMR dell'involucro di Vespa crabro (800MHz)}
  \label{fig:inv vespa crabro}
\end{figure}

\begin{figure}[H]
  \centering
  \includegraphics[width=1\textwidth]{VespC1D.png}
  \caption{Spettro ¹³C CP/MAS NMR delle celle di Vespula vulgaris (800MHz)}
  \label{fig:celle vespula}
\end{figure}

\begin{figure}[H]
  \centering
  \includegraphics[width=1\textwidth]{VespE1D.png}
  \caption{Spettro ¹³C CP/MAS NMR dell'involucro di Vespula vulgaris (800MHz)}
  \label{fig:involucro vespula}
\end{figure}

Quest'ultimo spettro è quello che ha destato un maggiore interesse nell'assegnazione dei segnali.
La componente cellulosica è molto più scarsa se confronta agli altri spettri, le bande che si osservano nella zona in blu sono anche dovute al segnale della chitina, e si osserva nella zona in cui si ha segnale proteico due bande distinte che non fanno riferimento a proteine.
Questi due segnali sono stati successivamente interpretati e sono stati assegnati a una componente lipidica (acidi grassi) e chitina. 

\subsection{Spettri bidimensionali}

Sono riportati gli spettri bidimensionali di Polistes dominula, Polistes gallicus e Vespula vulgaris; in questi sono state evidenziate tre differenti zone: in rosso la zona relativa al segnale del gruppo carbonilico delle proteine, in blu la zona relativa al segnale della cellulosa e in giallo la zona relativa all'altro segnale proteico.

\begin{figure}[H]
  \centering
  \includegraphics[width=1\textwidth]{Dom2D.png}
  \caption{Spettro bidimensionale Polistes dominula}
  \label{fig:Dom2D}
\end{figure}

\begin{figure}[H]
  \centering
  \includegraphics[width=1\textwidth]{Dom nuovo 700 0.31_2.51.png}
  \caption{Prima traccia del segnale del protone}
  \label{fig:Dom2D t1}
\end{figure}

\begin{figure}[H]
  \centering
  \includegraphics[width=1\textwidth]{Dom nuovo 700 2.22_5.84.png}
  \caption{Seconda traccia del segnale del protone Polistes dominula}
  \label{fig:Dom2D t2}
\end{figure}

\begin{figure}[H]
  \centering
  \includegraphics[width=1\textwidth]{Gal2D.png}
  \caption{Spettro bidimensionale Polistes gallicus}
  \label{fig:Gal2D}
\end{figure}

\begin{figure}[H]
  \centering
  \includegraphics[width=1\textwidth]{Gallicus nuovo 0.21_2.51.png}
  \caption{Prima traccia del segnale del protone Polistes gallicus}
  \label{fig:Dom2D t1}
\end{figure}

\begin{figure}[H]
  \centering
  \includegraphics[width=1\textwidth]{Gallicus nuovo 1.93_4.22.png}
  \caption{Seconda traccia del segnale del protone Polistes gallicus} 
  \label{fig:Dom2D t2}
\end{figure}

\begin{figure}[H]
  \centering
  \includegraphics[width=1\textwidth]{Vesp2D.png}
  \caption{Spettro bidimensionale Vespula vulgaris}
  \label{fig:Ves2D}
\end{figure}

\begin{figure}[H]
  \centering
  \includegraphics[width=1\textwidth]{Vespula esterno 3.21_6.93.png}
  \caption{Prima traccia del segnale del protone Vespula vulgaris}
  \label{fig:Ves2D t1}
\end{figure}

\begin{figure}[H]
  \centering
  \includegraphics[width=1\textwidth]{vespula esterno 6.35_9.33.png}
  \caption{Seconda traccia del segnale del protone Vespula vulgaris}
  \label{fig:VEs2D t2}
\end{figure}



