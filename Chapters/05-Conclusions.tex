% !TEX root = ../Thesis.tex

\chapter{Conclusioni e osservazioni finali}

Nel presente lavoro sono stati analizzati la composizione chimica e l’assetto strutturale dei nidi di diverse specie Polistes dominula, Polistes gallicus, Vespula vulgaris e Vespa crabro mediante spettroscopia NMR allo stato solido e valutazioni complementari.
L’insieme dei dati raccolti permette di evidenziare come specie anche morfologicamente o ecologicamente affini adottino strategie di nidificazione e scelte di materiali profondamente diverse, riflettendo adattamenti specifici al loro ambiente, alle abitudini di costruzione e ai vincoli selettivi che caratterizzano ciascuna specie.

Per le specie del genere Polistes le analisi NMR hanno mostrato, oltre all' elevata presenza di cellulosa, una grande eterogeneità strutturale relativa a sopratutto a quest'ultima; con segnali associati a porzioni cellulosiche amorfe e cristalline. 
Questa complessità molecolare e la relativa abbondanza di materiale proteico si traduce in un materiale leggero ma sorprendentemente resistente alla trazione e agli agenti climatici, caratteristiche in linea con le abitudini di nidificazione con il ruolo ecologico di queste specie.
Realizzando nidi in ambienti esposti a predatori e agenti atmosferici, considerando anche che queste specie non sviluppano involucri protettivi, si presenta la forte necessità di avere nidi nel complesso molto resistenti, influenzando quindi i materiali utilizzati

Un quadro molto diverso emerge per i generi Vespula e Vespa, che realizzano nidi chiusi e multilivello racchiusi da un involucro esterno. 
Anche per queste specie si osservano relazioni importanti tra la struttura nei nidi e caratteristiche sociali.
Per le celle e involucro esterno di Vespa crabro non si osserva una particolare distinzione a livello di composizione, i nidi realizzati da queste specie sono poi in genere molto più fragili rispetto agli altri analizzati.
Durante la preparazione del campione si è osservata la facilità con cui si riusciva a sbriciolare il nido, questa relativa fragilità è data dalla scelta dei siti in cui nidificare, protetti o poco esposti, questo e il fatto che Vespa crabro non si deve difendere da predatori fa si che non ci sia una marcata esigenza nella costruzione di favi resistenti.
Per Vespula vulgaris il discorso è differente, anche loro realizzano nidi in luoghi non esposti con celle poco resistenti; tuttavia il fatto che queste specie nidificano sottoterra fa si che l'involucro debba avere una determinata composizione.
Si è osservato infatti una nitevole differenza nella struttura tra le celle interne e l'esterno, con un involucro ricco di chitina e molto robusto, se confrontato con le celle o con l'involucro di Vespa crabro.

Le simulazioni delle possibili interazioni tra cellulosa e proteine suggeriscono un'organizzazione del materiale in cui domini cellulosici possono associarsi a superfici proteiche tramite interazioni deboli, contribuendo alla stabilità del nido. 
Sebbene preliminari, queste analisi forniscono indicazioni utili per comprendere come le componenti organiche possano cooperare nel determinare le proprietà macroscopiche della carta delle vespe.

In conclusione, questo studio mette in evidenza come la chimica del materiale di costruzione e la biologia delle specie siano strettamente interconnesse: la microstruttura riflette l’ecologia della specie, e la selezione del materiale, così come la sua rielaborazione, rappresentano un tratto adattativo fondamentale nella storia evolutiva dei Vespidae. 
L’approccio integrato basato sulla spettroscopia a stato solido si dimostra quindi uno strumento efficace per investigare questi biocompositi naturali.